\documentclass[11pt]{article}

\usepackage[utf8]{inputenc}
\usepackage[T1]{fontenc}
\usepackage{graphicx}
\usepackage{longtable}
\usepackage{wrapfig}
\usepackage{rotating}
\usepackage[normalem]{ulem}
\usepackage{amsmath}
\usepackage{amssymb}
\usepackage{capt-of}
\usepackage{hyperref}
\usepackage[margin=0.75in]{geometry}
\usepackage{ragged2e}
\usepackage{mdframed}
\usepackage{shapepar}
\date{}
\title{\textbf{TITOLO}}
\begin{document}
\begin{center}
\huge\textbf{TITOLO}
\end{center}
\begin{center}
eventuali note


\end{center}
\begin{mdframed}
\begin{center}
\textbf{Istruzioni Generali}
\end{center}
\begin{itemize}
\item Per ogni problema, indicare sul cartellino delle risposte un intero compreso tra 0000 e 9999.
\item Se la quantità richiesta non è un numero intero, dove non indicato diversamente, si indichi la sua parte intera.
\item Se la quantità richiesta è un numero negativo, oppure se il problema non ha soluzione, si indichi 0000.
\item Se la quantità richiesta è maggiore di 9999, si indichino le ultime quattro cifre della sua parte intera.
\item Nello svolgimento dei calcoli può essere utile tener conto dei seguenti valori approssimati:
\end{itemize}

\begin{center}
$\sqrt2 = 1,4142 \qquad \qquad \sqrt3 = 1,7321 \qquad \qquad \sqrt5 = 2,2360 \qquad \qquad \pi = 3,1415$
\end{center}
\end{mdframed}

\section{Titolo problema\hfill\normalsize\normalfont \emph{(Nome autore testo)}}
Testo problema
\end{document}
